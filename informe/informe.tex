\documentclass[10pt,letterpaper]{article}

\usepackage{listings}
\lstloadlanguages{[ANSI]C}
\usepackage{amsmath}
\usepackage{amssymb}
\usepackage{graphicx}
\usepackage[spanish,activeacute]{babel}
\usepackage{setspace}
\onehalfspacing

\author{Julio Castillo \ 02-34745\\C\'esar Romero 02-35409}
\title{CI4321\\Marbles: Reporte T\'ecnico}

\begin{document}
\maketitle{}
\section{OpenGL}
Para la implementaci'on de este juego era esencial utilizar una
superficie de B'ezier que representa el terreno de juego sobre el
cual se mueven las metras. OpenGL se encarga de hacer los c'alculos
necesarios para dibujar la superficie incluyendo todas sus normales de
manera que la interacci'on con la luz sea correcta. Esto simplifica
parte importante de la implementaci'on. Con s'olo especificar los
puntos de control podemos crear una superficie y modificarla.

La otra parte importante en la implementaci'on del juego la
constituyen las metras, su interacci'on con la superficie y con otras
metras. A pesar de que la evaluaci'on de la superficie es realizada en
su totalidad por OpenGL, no es posible para los programadores tener
acceso a los valores de esta funci'on. Esto nos lleva a la necesidad
de tener que hacer estos c'alculos por separado, lo que sugiere que
realmente se est'an realizando algunas cuentas dos veces.

Tanto la evaluaci'on de la superficie como de la normal son realizados
en nuestra implementaci'on y ser'an explicados a continuaci'on.

\section{B'ezier-Spline}
Para modelar el comportamiento de las metras sobre el terreno de juego
es necesario evaluar constantemente la superficie de B'ezier para
obtener la ``altura'' a la que van a ser dibujadas. La superficie
est'a definida en funci'on de un conjunto de $(n+1)(m+1)$ puntos de
control. La f'ormula para la superficie~\cite{angel} recibe como par'ametro puntos
de un cuadrado unitario $(u,v) \in [0,1]^2$
\[
p(u,v)=\sum_{i=0}^n\sum_{j=0}^mB_i^n(u)B_j^m(v)k_{ij}
\]
donde
\[
B_i^n(u)={n\choose i}u^i(1-u)^{n-i}
\]
es un polinomio de Bernstein.

Adicionalmente es necesario calcular el vector normal al plano en un
punto $(u_0,v_0)$. Para esto tomamos un par de puntos con diferencia
$0.5$ en $x$ y $z$ respectivamente, construimos dos vectores que van
desde el punto sobre la superficie hacia cada uno de estos puntos y
con el producto cruz de estos obtenemos el vector normal.

No es dif'icil ver que el vector resultante no es de magnitud 1. Es
mas, la magnitud de este vector resultante es directamente
proporcional a la inclinaci'on de la superficie en el 'area cercana al
punto considerado. Esto nos da un modelo simple que permite tomar las
componentes del vector resultante como valores que definen la
velocidad de la metra en cualquier momento que no ocurra una colisi'on.

Para poder conocer la posici'on de una metra en un instante de tiempo
$t$ es necesario terminar de definir el modelo. Tenemos todav'ia tres
posibles escenarios por definir. Primero, si no hay colisi'on la
velocidad resultante de la metra se calcula usando tanto la velocidad
actual como la velocidad en un instante de tiempo previo
$V=0.85*(V_0+V_1)$, donde $V_0$ y $V_1$ representan la velocidad de la
metra en el instante $t-\Delta t$ y $t$, respectivamente. N'otese que
esto hace que la metra acelere si se encuentra en una superficie inclinada y
que desacelere si se encuentra en una plana. Segundo, si la metra $M_i$
choca con $M_j$, se intercambian las velocidades de las metras:
$V_{Mi},V_{Mj}:=V_{Mj},V_{Mi}$.

Por 'ultimo, para describir el choque con la ``pared invisible''\ que
delimita el terreno de juego simplemente cambiabos el signo del
componente en $X$ o $Z$, si la metra llega a un l'imite en $X$ o $Z$
repectivamente, y lo multiplicamos por $0.5$ para simular la p'erdida
de energ'ia.
\[
V_{x(z)}=
\begin{cases}
-0.5\times V_{x(z)} & \text{si }|M_{x(z)}| > 5\\
V_{x(z)}            &\text{si } |M_{x(z)}| < 5\\
\end{cases}
\]
\section{Estructuras de Datos}
Se utilizaron dos estructuras de datos principales. La primera y m'as
importante representa el modelo completo del juego. Una instancia de
esta estructura se maneja durante toda la corrida del programa. A
continuaci'on, su definici'on:
\lstset{language=[ANSI]C,tabsize=2}
\begin{lstlisting}

  typedef struct {
    
    int editando;             // estamos editando?
    int editx;                // punto de edicion
    int edity;

    int gridsize;             // resolucion

    GLfloat puntos[11][11][3];// puntos de control

    GLfloat camx, camy;       // angulo de la camara en x, y
    GLfloat camz;             // distancia de la camara al centro
    GLfloat camaperture;      // apertura de la camara

    int winw;                 // anchura de la ventana
    int winh;                 // altura de la ventana

    int wireframe;            // fill o wireframe

    float objx, objy, objz;   // posicion de la metra objetivo
    int ronda;                // ronda del juego
    int over;                 // game over?

    int ejes;                 // dibujar ejes?
    
    int players;              // numero de jugadores
    int turn;                 // jugador de turno
    marble_t marble[4];       // maximo 4 jugadorese

    int still;                // no hay metras moviendose
    int wait_still;    // esperando para empezar el proximo turno


  } model_t;

\end{lstlisting}

Como vemos, parte de nuestro modelo est'a constituido por un arreglo de
\verb|marble_t| que contiene la informaci'on de todas las metras en el
terreno de juego. La segunda estructura es precisamente la que define
las metras y todas las caracteristicas que sean requeridas para
representarlas correctamente. Se define de la siguiente manera:
\begin{lstlisting}

  typedef struct{

    float x,y,z;  // posicion de la metra
    float r;      // radio 

    float vx;     // velocidad en X
    float vz;     // velocidad en Z

    float nx;     // componente en X del vector normal
    float ny;     // componente en Y
    float nz;     // componente en Z

    material_t material;

    GLUquadricObj *q;
    GLuint texture;

    GLfloat rotx;  // rotacion en X
    GLfloat rotz;  // rotacion en Z

    GLfloat theta; // angulo del disparar (en radianes)
    GLfloat mag;   // magnitud del vector de fuerza de disparo

  } marble_t;

\end{lstlisting}
\section{Juego}
Se implement'o un juego de metras donde el objetivo del jugador es
acercar su metra lo mas posible a una esfera peque'na de color negro
que es colocada en una posici'on aleatoria del terreno del juego antes
de iniciar la partida. Cada jugador tendra dos turnos en los cuales
puede acercarse y alejar a sus contendientes del objetivo, pero el
objetivo siempre permanece en la misma posici'on.

El terreno de juego es inicialmente plano pero puede ser editado en
cualquier momento durante la partida
\section{Problemas enfrentados}
El primer problema de importancia que enfrentamos durante la
implementaci'on de este juego fue con la evaluaci'on de la superficie
para poder dibujar las metras en el lugar adecuado. Al implementar la
f'ormula que define el B'ezier-Spline, decidimos inicialmente usar de
referencia el Red Book de OpenGL\cite{redbook}. En la explicaci'on de
la f'ormula que define la superficie encontramos un error relacionado
con el n'umero de puntos de control utilizados en el c'alculo. El
problema fue resuelto luego de consultar otras fuentes entre las
cuales se encuentra la Wikipedia\cite{wikibezier}.

Otro problema importante fue con el dibujado del spline. Encontramos
que al correr nuestro juego en tarjetas NVIDIA el spline s'olo se
dibuja correctamente cuando tomamos un n'umero de puntos de control
menor o igual a 64. Lamentablemente este problema no pudo ser
resuelto. En todas las tarjetas gr'aficas de otras marcas que fueron
probadas, el juego corri'o sin problemas.

\begin{thebibliography}{99}
\bibitem{redbook} Jackie Neider, Tom Davis, and Mason Woo. OpenGL
  Programming Guide. Addison-Wesley.
\bibitem{wikibezier} \verb|http://en.wikipedia.org/wiki/B'ezier_surface|
\bibitem{angel}Edward Angel. Interactive Computer Graphics. Addison-Wesley Longman Publishing Co., Inc. Boston, MA, USA. 2002
\end{thebibliography}
\end{document}
